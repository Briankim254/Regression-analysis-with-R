% Options for packages loaded elsewhere
\PassOptionsToPackage{unicode}{hyperref}
\PassOptionsToPackage{hyphens}{url}
%
\documentclass[
]{article}
\usepackage{amsmath,amssymb}
\usepackage{lmodern}
\usepackage{iftex}
\ifPDFTeX
  \usepackage[T1]{fontenc}
  \usepackage[utf8]{inputenc}
  \usepackage{textcomp} % provide euro and other symbols
\else % if luatex or xetex
  \usepackage{unicode-math}
  \defaultfontfeatures{Scale=MatchLowercase}
  \defaultfontfeatures[\rmfamily]{Ligatures=TeX,Scale=1}
\fi
% Use upquote if available, for straight quotes in verbatim environments
\IfFileExists{upquote.sty}{\usepackage{upquote}}{}
\IfFileExists{microtype.sty}{% use microtype if available
  \usepackage[]{microtype}
  \UseMicrotypeSet[protrusion]{basicmath} % disable protrusion for tt fonts
}{}
\makeatletter
\@ifundefined{KOMAClassName}{% if non-KOMA class
  \IfFileExists{parskip.sty}{%
    \usepackage{parskip}
  }{% else
    \setlength{\parindent}{0pt}
    \setlength{\parskip}{6pt plus 2pt minus 1pt}}
}{% if KOMA class
  \KOMAoptions{parskip=half}}
\makeatother
\usepackage{xcolor}
\usepackage[margin=1in]{geometry}
\usepackage{color}
\usepackage{fancyvrb}
\newcommand{\VerbBar}{|}
\newcommand{\VERB}{\Verb[commandchars=\\\{\}]}
\DefineVerbatimEnvironment{Highlighting}{Verbatim}{commandchars=\\\{\}}
% Add ',fontsize=\small' for more characters per line
\usepackage{framed}
\definecolor{shadecolor}{RGB}{248,248,248}
\newenvironment{Shaded}{\begin{snugshade}}{\end{snugshade}}
\newcommand{\AlertTok}[1]{\textcolor[rgb]{0.94,0.16,0.16}{#1}}
\newcommand{\AnnotationTok}[1]{\textcolor[rgb]{0.56,0.35,0.01}{\textbf{\textit{#1}}}}
\newcommand{\AttributeTok}[1]{\textcolor[rgb]{0.77,0.63,0.00}{#1}}
\newcommand{\BaseNTok}[1]{\textcolor[rgb]{0.00,0.00,0.81}{#1}}
\newcommand{\BuiltInTok}[1]{#1}
\newcommand{\CharTok}[1]{\textcolor[rgb]{0.31,0.60,0.02}{#1}}
\newcommand{\CommentTok}[1]{\textcolor[rgb]{0.56,0.35,0.01}{\textit{#1}}}
\newcommand{\CommentVarTok}[1]{\textcolor[rgb]{0.56,0.35,0.01}{\textbf{\textit{#1}}}}
\newcommand{\ConstantTok}[1]{\textcolor[rgb]{0.00,0.00,0.00}{#1}}
\newcommand{\ControlFlowTok}[1]{\textcolor[rgb]{0.13,0.29,0.53}{\textbf{#1}}}
\newcommand{\DataTypeTok}[1]{\textcolor[rgb]{0.13,0.29,0.53}{#1}}
\newcommand{\DecValTok}[1]{\textcolor[rgb]{0.00,0.00,0.81}{#1}}
\newcommand{\DocumentationTok}[1]{\textcolor[rgb]{0.56,0.35,0.01}{\textbf{\textit{#1}}}}
\newcommand{\ErrorTok}[1]{\textcolor[rgb]{0.64,0.00,0.00}{\textbf{#1}}}
\newcommand{\ExtensionTok}[1]{#1}
\newcommand{\FloatTok}[1]{\textcolor[rgb]{0.00,0.00,0.81}{#1}}
\newcommand{\FunctionTok}[1]{\textcolor[rgb]{0.00,0.00,0.00}{#1}}
\newcommand{\ImportTok}[1]{#1}
\newcommand{\InformationTok}[1]{\textcolor[rgb]{0.56,0.35,0.01}{\textbf{\textit{#1}}}}
\newcommand{\KeywordTok}[1]{\textcolor[rgb]{0.13,0.29,0.53}{\textbf{#1}}}
\newcommand{\NormalTok}[1]{#1}
\newcommand{\OperatorTok}[1]{\textcolor[rgb]{0.81,0.36,0.00}{\textbf{#1}}}
\newcommand{\OtherTok}[1]{\textcolor[rgb]{0.56,0.35,0.01}{#1}}
\newcommand{\PreprocessorTok}[1]{\textcolor[rgb]{0.56,0.35,0.01}{\textit{#1}}}
\newcommand{\RegionMarkerTok}[1]{#1}
\newcommand{\SpecialCharTok}[1]{\textcolor[rgb]{0.00,0.00,0.00}{#1}}
\newcommand{\SpecialStringTok}[1]{\textcolor[rgb]{0.31,0.60,0.02}{#1}}
\newcommand{\StringTok}[1]{\textcolor[rgb]{0.31,0.60,0.02}{#1}}
\newcommand{\VariableTok}[1]{\textcolor[rgb]{0.00,0.00,0.00}{#1}}
\newcommand{\VerbatimStringTok}[1]{\textcolor[rgb]{0.31,0.60,0.02}{#1}}
\newcommand{\WarningTok}[1]{\textcolor[rgb]{0.56,0.35,0.01}{\textbf{\textit{#1}}}}
\usepackage{graphicx}
\makeatletter
\def\maxwidth{\ifdim\Gin@nat@width>\linewidth\linewidth\else\Gin@nat@width\fi}
\def\maxheight{\ifdim\Gin@nat@height>\textheight\textheight\else\Gin@nat@height\fi}
\makeatother
% Scale images if necessary, so that they will not overflow the page
% margins by default, and it is still possible to overwrite the defaults
% using explicit options in \includegraphics[width, height, ...]{}
\setkeys{Gin}{width=\maxwidth,height=\maxheight,keepaspectratio}
% Set default figure placement to htbp
\makeatletter
\def\fps@figure{htbp}
\makeatother
\setlength{\emergencystretch}{3em} % prevent overfull lines
\providecommand{\tightlist}{%
  \setlength{\itemsep}{0pt}\setlength{\parskip}{0pt}}
\setcounter{secnumdepth}{-\maxdimen} % remove section numbering
\ifLuaTeX
  \usepackage{selnolig}  % disable illegal ligatures
\fi
\IfFileExists{bookmark.sty}{\usepackage{bookmark}}{\usepackage{hyperref}}
\IfFileExists{xurl.sty}{\usepackage{xurl}}{} % add URL line breaks if available
\urlstyle{same} % disable monospaced font for URLs
\hypersetup{
  pdftitle={Multiple Regression Analysis Case Study},
  pdfauthor={A. Wagala},
  hidelinks,
  pdfcreator={LaTeX via pandoc}}

\title{Multiple Regression Analysis Case Study}
\author{A. Wagala}
\date{2023-01-28}

\begin{document}
\maketitle

\hypertarget{introduction}{%
\subsection{Introduction}\label{introduction}}

In this case study, we use the Base Ball Players Data using the
following Steps

\hypertarget{step-1-data-collection}{%
\subsubsection{Step 1: Data Collection}\label{step-1-data-collection}}

We utilize the MLB data ``MLB\_data.txt''. It contains 1034 records of
heights and weights for some current and recent Major League Baseball
(MLB) Players. This dataset includes the following variables:

\begin{itemize}
\tightlist
\item
  Name: MLB Player Name,
\item
  Team: The Baseball team the player was a member of at the time the
  data was acquired,
\item
  Position: Player field position,
\item
  Height: Player height in inch,
\item
  Weight: Player weight in pounds, and
\item
  Age: Player age at time of record.
\end{itemize}

This is an R Markdown presentation. Markdown is a simple formatting
syntax for authoring HTML, PDF, and MS Word documents. For more details
on using R Markdown see \url{http://rmarkdown.rstudio.com}.

When you click the \textbf{Knit} button a document will be generated
that includes both content as well as the output of any embedded R code
chunks within the document.

\hypertarget{step-2-exploring-and-preparing-the-data}{%
\subsection{Step 2: Exploring and Preparing the
Data}\label{step-2-exploring-and-preparing-the-data}}

Load the data

\begin{Shaded}
\begin{Highlighting}[]
\CommentTok{\#Load the data}
\NormalTok{mlb }\OtherTok{\textless{}{-}} \FunctionTok{read.delim}\NormalTok{(}\StringTok{"C:/Users/kimut/Downloads/regression aalysis/MLB\_data.txt"}\NormalTok{)}
\FunctionTok{str}\NormalTok{(mlb)}
\end{Highlighting}
\end{Shaded}

\begin{verbatim}
## 'data.frame':    1034 obs. of  6 variables:
##  $ Name    : chr  "Adam_Donachie " "Paul_Bako " "Ramon_Hernandez " "Kevin_Millar " ...
##  $ Team    : chr  "BAL " "BAL " "BAL " "BAL " ...
##  $ Position: chr  "Catcher " "Catcher " "Catcher " "First_Baseman " ...
##  $ Height  : num  74 74 72 72 73 69 69 71 76 71 ...
##  $ Weight  : num  180 215 210 210 188 176 209 200 231 180 ...
##  $ Age     : num  23 34.7 30.8 35.4 35.7 ...
\end{verbatim}

\begin{Shaded}
\begin{Highlighting}[]
\NormalTok{mlb}\OtherTok{\textless{}{-}}\NormalTok{mlb[, }\SpecialCharTok{{-}}\DecValTok{1}\NormalTok{]}
\end{Highlighting}
\end{Shaded}

The output shows that the variable TEAM and Position are misspecified as
characters. We can fix this by using the function as. factor() to
convert numerical or character vectors to factors.

\begin{Shaded}
\begin{Highlighting}[]
\NormalTok{mlb}\SpecialCharTok{$}\NormalTok{Team}\OtherTok{\textless{}{-}}\FunctionTok{as.factor}\NormalTok{(mlb}\SpecialCharTok{$}\NormalTok{Team)}
\NormalTok{mlb}\SpecialCharTok{$}\NormalTok{Position}\OtherTok{\textless{}{-}}\FunctionTok{as.factor}\NormalTok{(mlb}\SpecialCharTok{$}\NormalTok{Position)}
\end{Highlighting}
\end{Shaded}

We can explore the data by starting with the basic statistics

\begin{Shaded}
\begin{Highlighting}[]
\FunctionTok{summary}\NormalTok{(mlb}\SpecialCharTok{$}\NormalTok{Weight)}
\end{Highlighting}
\end{Shaded}

\begin{verbatim}
##    Min. 1st Qu.  Median    Mean 3rd Qu.    Max. 
##   150.0   187.0   200.0   201.7   215.0   290.0
\end{verbatim}

\begin{Shaded}
\begin{Highlighting}[]
\FunctionTok{hist}\NormalTok{(mlb}\SpecialCharTok{$}\NormalTok{Weight, }\AttributeTok{main =} \StringTok{"Histogram for Weights"}\NormalTok{)}
\end{Highlighting}
\end{Shaded}

\includegraphics{KABU_CaseStudy1_files/figure-latex/unnamed-chunk-3-1.pdf}

The plot shows that this distribution appears somewhat right-skewed.

\begin{itemize}
\tightlist
\item
  We can do further exploration by to obtaining a compact dataset
  summary where we can mark heavy weight and light weight players
  (according to light \textless{} median \textless{} heavy) by different
  colors in the plot.
\end{itemize}

\begin{Shaded}
\begin{Highlighting}[]
\CommentTok{\#Pair plots of the MLB data by player’s light (red) or heavy (blue) weights}
\FunctionTok{require}\NormalTok{(GGally)}
\end{Highlighting}
\end{Shaded}

\begin{verbatim}
## Loading required package: GGally
\end{verbatim}

\begin{verbatim}
## Loading required package: ggplot2
\end{verbatim}

\begin{verbatim}
## Registered S3 method overwritten by 'GGally':
##   method from   
##   +.gg   ggplot2
\end{verbatim}

\begin{Shaded}
\begin{Highlighting}[]
\NormalTok{mlb\_binary }\OtherTok{=}\NormalTok{ mlb}
\NormalTok{mlb\_binary}\SpecialCharTok{$}\NormalTok{bi\_weight }\OtherTok{=}
  \FunctionTok{as.factor}\NormalTok{(}\FunctionTok{ifelse}\NormalTok{(mlb\_binary}\SpecialCharTok{$}\NormalTok{Weight}\SpecialCharTok{\textgreater{}}\FunctionTok{median}\NormalTok{(mlb\_binary}\SpecialCharTok{$}\NormalTok{Weight),}\DecValTok{1}\NormalTok{,}\DecValTok{0}\NormalTok{))}
\NormalTok{g\_weight }\OtherTok{\textless{}{-}} \FunctionTok{ggpairs}\NormalTok{(}\AttributeTok{data=}\NormalTok{mlb\_binary[}\SpecialCharTok{{-}}\DecValTok{1}\NormalTok{], }\AttributeTok{title=}\StringTok{"MLB Light/Heavy Weights"}\NormalTok{,}
                    \AttributeTok{mapping=}\NormalTok{ggplot2}\SpecialCharTok{::}\FunctionTok{aes}\NormalTok{(}\AttributeTok{colour =}\NormalTok{ bi\_weight),}
                    \AttributeTok{lower=}\FunctionTok{list}\NormalTok{(}\AttributeTok{combo=}\FunctionTok{wrap}\NormalTok{(}\StringTok{"facethist"}\NormalTok{,}\AttributeTok{binwidth=}\DecValTok{1}\NormalTok{)))}
\NormalTok{g\_weight}
\end{Highlighting}
\end{Shaded}

\includegraphics{KABU_CaseStudy1_files/figure-latex/unnamed-chunk-4-1.pdf}

-We can also mark player positions by different colors in the plot.

\begin{Shaded}
\begin{Highlighting}[]
\NormalTok{g\_position }\OtherTok{\textless{}{-}} \FunctionTok{ggpairs}\NormalTok{(}\AttributeTok{data=}\NormalTok{mlb[}\SpecialCharTok{{-}}\DecValTok{1}\NormalTok{], }\AttributeTok{title=}\StringTok{"MLB by Position"}\NormalTok{,}
                      \AttributeTok{mapping=}\NormalTok{ggplot2}\SpecialCharTok{::}\FunctionTok{aes}\NormalTok{(}\AttributeTok{colour =}\NormalTok{ Position),}
                      \AttributeTok{lower=}\FunctionTok{list}\NormalTok{(}\AttributeTok{combo=}\FunctionTok{wrap}\NormalTok{(}\StringTok{"facethist"}\NormalTok{,}\AttributeTok{binwidth=}\DecValTok{1}\NormalTok{)))}
\NormalTok{g\_position}
\end{Highlighting}
\end{Shaded}

\includegraphics{KABU_CaseStudy1_files/figure-latex/unnamed-chunk-5-1.pdf}

We now explore the predictors

\begin{Shaded}
\begin{Highlighting}[]
\FunctionTok{table}\NormalTok{(mlb}\SpecialCharTok{$}\NormalTok{Team)}
\end{Highlighting}
\end{Shaded}

\begin{verbatim}
## 
## ANA  ARZ  ATL  BAL  BOS  CHC  CIN  CLE  COL  CWS  DET  FLA  HOU   KC   LA  MIN  
##   35   28   37   35   36   36   36   35   35   33   37   32   34   35   33   33 
## MLW  NYM  NYY  OAK  PHI  PIT   SD  SEA   SF  STL   TB  TEX  TOR  WAS  
##   35   38   32   37   36   35   33   34   34   32   33   35   34   36
\end{verbatim}

\begin{Shaded}
\begin{Highlighting}[]
\FunctionTok{table}\NormalTok{(mlb}\SpecialCharTok{$}\NormalTok{Position)}
\end{Highlighting}
\end{Shaded}

\begin{verbatim}
## 
##           Catcher  Designated_Hitter      First_Baseman         Outfielder  
##                 76                 18                 55                194 
##    Relief_Pitcher     Second_Baseman          Shortstop   Starting_Pitcher  
##                315                 58                 52                221 
##     Third_Baseman  
##                 45
\end{verbatim}

\begin{Shaded}
\begin{Highlighting}[]
\FunctionTok{summary}\NormalTok{(mlb}\SpecialCharTok{$}\NormalTok{Height)}
\end{Highlighting}
\end{Shaded}

\begin{verbatim}
##    Min. 1st Qu.  Median    Mean 3rd Qu.    Max. 
##    67.0    72.0    74.0    73.7    75.0    83.0
\end{verbatim}

\begin{Shaded}
\begin{Highlighting}[]
\FunctionTok{summary}\NormalTok{(mlb}\SpecialCharTok{$}\NormalTok{Age)}
\end{Highlighting}
\end{Shaded}

\begin{verbatim}
##    Min. 1st Qu.  Median    Mean 3rd Qu.    Max. 
##   20.90   25.44   27.93   28.74   31.23   48.52
\end{verbatim}

In this case, we have two numerical predictors, two categorical
predictors and 1,034 observations.

\hypertarget{exploring-relationships-among-features-the-correlation-matrix}{%
\subsubsection{Exploring Relationships Among Features: The Correlation
Matrix}\label{exploring-relationships-among-features-the-correlation-matrix}}

\begin{itemize}
\tightlist
\item
  Before fitting a model, its important to examine the independence of
  the potential predictors and the dependent variable.
\item
  Multiple linear regression assumes that predictors are all independent
  of each other.
\end{itemize}

\begin{Shaded}
\begin{Highlighting}[]
\FunctionTok{cor}\NormalTok{(mlb[}\FunctionTok{c}\NormalTok{(}\StringTok{"Weight"}\NormalTok{, }\StringTok{"Height"}\NormalTok{, }\StringTok{"Age"}\NormalTok{)])}
\end{Highlighting}
\end{Shaded}

\begin{verbatim}
##           Weight      Height         Age
## Weight 1.0000000  0.53031802  0.15784706
## Height 0.5303180  1.00000000 -0.07367013
## Age    0.1578471 -0.07367013  1.00000000
\end{verbatim}

\begin{itemize}
\tightlist
\item
  This looks very good and wouldn't cause any multicollinearity problem.
\item
  If two of our predictors are highly correlated, they both provide
  almost the same information, which could implymulticollinearity.
\item
  A common practice is to delete one of them in the model or use
  dimensionality reduction methods. Furthermore, we can visualize the
  correlation as follows
\end{itemize}

\begin{Shaded}
\begin{Highlighting}[]
\FunctionTok{pairs}\NormalTok{(mlb[}\FunctionTok{c}\NormalTok{(}\StringTok{"Weight"}\NormalTok{, }\StringTok{"Height"}\NormalTok{, }\StringTok{"Age"}\NormalTok{)])}
\end{Highlighting}
\end{Shaded}

\includegraphics{KABU_CaseStudy1_files/figure-latex/unnamed-chunk-8-1.pdf}

The above plots do not give a clear sense of the linearity of the data.
We can use other packages to get a better feel.

\begin{Shaded}
\begin{Highlighting}[]
\FunctionTok{library}\NormalTok{(psych)}
\end{Highlighting}
\end{Shaded}

\begin{verbatim}
## 
## Attaching package: 'psych'
\end{verbatim}

\begin{verbatim}
## The following objects are masked from 'package:ggplot2':
## 
##     %+%, alpha
\end{verbatim}

\begin{Shaded}
\begin{Highlighting}[]
\FunctionTok{pairs.panels}\NormalTok{(mlb[, }\FunctionTok{c}\NormalTok{(}\StringTok{"Weight"}\NormalTok{, }\StringTok{"Height"}\NormalTok{, }\StringTok{"Age"}\NormalTok{)])}
\end{Highlighting}
\end{Shaded}

\includegraphics{KABU_CaseStudy1_files/figure-latex/unnamed-chunk-9-1.pdf}

\begin{itemize}
\tightlist
\item
  The diagonal, we have our correlation coefficients in numerical form.
\item
  On the diagonal, there are histograms of variables.
\item
  Below the diagonal, visual information is presented to help us
  understand the trend.
\item
  This specific graph shows that height and weight are positively and
  strongly correlated.
\item
  The relationships between age and height, as well as, age and weight
  are very weak.
\item
  the horizontal red line in the panel below the main diagonal graphs,
  which indicates weak relationship
\end{itemize}

\hypertarget{step-3-training-a-model-on-the-data}{%
\subsubsection{Step 3: Training a Model on the
Data}\label{step-3-training-a-model-on-the-data}}

Use the codes

\begin{Shaded}
\begin{Highlighting}[]
\NormalTok{fit}\OtherTok{\textless{}{-}}\FunctionTok{lm}\NormalTok{(Weight}\SpecialCharTok{\textasciitilde{}}\NormalTok{Height}\SpecialCharTok{+}\NormalTok{Age, }\AttributeTok{data=}\NormalTok{mlb)}
\NormalTok{fit}
\end{Highlighting}
\end{Shaded}

\begin{verbatim}
## 
## Call:
## lm(formula = Weight ~ Height + Age, data = mlb)
## 
## Coefficients:
## (Intercept)       Height          Age  
##   -191.6695       4.9626       0.9624
\end{verbatim}

\hypertarget{step-4-evaluating-model-performance}{%
\subsubsection{Step 4: Evaluating Model
Performance}\label{step-4-evaluating-model-performance}}

We examine the model performance

\begin{Shaded}
\begin{Highlighting}[]
\FunctionTok{summary}\NormalTok{(fit)}
\end{Highlighting}
\end{Shaded}

\begin{verbatim}
## 
## Call:
## lm(formula = Weight ~ Height + Age, data = mlb)
## 
## Residuals:
##     Min      1Q  Median      3Q     Max 
## -50.818 -12.148  -0.344  10.720  74.175 
## 
## Coefficients:
##              Estimate Std. Error t value Pr(>|t|)    
## (Intercept) -191.6695    17.9172  -10.70  < 2e-16 ***
## Height         4.9626     0.2345   21.16  < 2e-16 ***
## Age            0.9624     0.1252    7.69 3.43e-14 ***
## ---
## Signif. codes:  0 '***' 0.001 '**' 0.01 '*' 0.05 '.' 0.1 ' ' 1
## 
## Residual standard error: 17.33 on 1031 degrees of freedom
## Multiple R-squared:  0.3202, Adjusted R-squared:  0.3189 
## F-statistic: 242.8 on 2 and 1031 DF,  p-value: < 2.2e-16
\end{verbatim}

\begin{Shaded}
\begin{Highlighting}[]
\FunctionTok{plot}\NormalTok{(fit, }\AttributeTok{which =} \DecValTok{1}\SpecialCharTok{:}\DecValTok{2}\NormalTok{)}
\end{Highlighting}
\end{Shaded}

\includegraphics{KABU_CaseStudy1_files/figure-latex/unnamed-chunk-11-1.pdf}
\includegraphics{KABU_CaseStudy1_files/figure-latex/unnamed-chunk-11-2.pdf}

\hypertarget{note}{%
\paragraph{Note:}\label{note}}

\begin{itemize}
\item
  The model summary shows us how well the model fits the data.
\item
  Residuals: This tells us about the residuals. If we have extremely
  large or extremely small residuals for some observations compared to
  the rest of residuals, either they are outliers due to reporting error
  or the model fits data poorly. Check the minimum and the maximum
  residuals.
\item
  The residuals could be characterized by examining their range and by
  viewing the residual diagnostic plots.
\item
  Coefficients: In this section of the output, we look at the very right
  column that has symbols like stars or dots showing if that variable is
  significant and should be included in the model.
\item
  However, if no symbol is included next to a variable, then it means
  this estimated covariate coefficient in the linear model covariance
  could be trivial.
\item
  Another thing we can look at is the Pr(\textgreater\textbar t\textbar)
  column. A number close to zero in this column indicates the row
  variable is significant, otherwise it could be removed from the model.
  Here, both Age and Height are significant.
\item
  R-squared What percent in \(y\) is explained by the included
  predictors?
\item
  A well-fitted linear regression would have R-squared over 70\%.
\item
  The diagnostic plots also help us understand the model quality.
  Residual vs.~Fitted This is the main residual diagnostic plot, we can
  check ouliers.
\item
  Normal Q-Q plot examines the normality assumption of the model. The
  scattered dots represent the matched quantiles of the data and the
  normal distribution. If the Q-Q plot closely resembles a line
  bisecting the first quadrant in the plane, the normality assumption is
  valid.
\end{itemize}

\hypertarget{step-5-improving-model-performance}{%
\subsubsection{Step 5: Improving Model
Performance}\label{step-5-improving-model-performance}}

\begin{itemize}
\tightlist
\item
  We can perform forward or backward selection of important
  features/predictors.
\end{itemize}

-In most cases,backward-selection is preferable because it tends to
retain much larger models.

-There are various criteria that can be used to evaluate a model.

\hypertarget{assignment}{%
\subsubsection{Assignment}\label{assignment}}

Use the \emph{Symptom\_ChronicIllness} Data Set to fit several different
Multiple Linear Regression models predicting clinically relevant
outcomes.

\end{document}
